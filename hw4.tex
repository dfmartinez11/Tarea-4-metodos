\documentclass[13pts]{article}
\usepackage[pdftex]{graphicx}
\author{Diego Arguello}
\title{Especificaciones de la TAREA 4}
\begin{document}
\maketitle
\begin{abstract}
En este corto documento muestro las graficas del movimiento de una particula en un campo gravitacional y en un medio que genera fuerzas resitivas proporcionales al cuadrado de la velocidad. Tambien muestro las distribuciones de temperatura y su valor promedio, ambas segun una evolucion temporal para una seccion transversal de una roca con una barilla en su interior a cierta temperatura constante.
\end{abstract}

\section{Objeto en campo gravitacional de aceleracion g y medio resistivo}

\begin{figure}
	\centering
	\includegraphics{Velocidad_40.png}
	\caption[scale=0.01]{Caracteristicas de velocidad con agnulo = 40 grados}
	\label{40}
\end{figure}

\begin{figure}
	\centering
	\includegraphics[scale=0.4]{trayectoria.png}
	\caption{Movimiento con mismas velocidades pero angulos distintos}
	\label{trayect}
\end{figure}


\section{Difusion de la temperatura en la seccion transversal cuadrada de una roca}
blalablabla

\begin{figure}
	\centering
	\includegraphics[scale=0.35]{T_promedios.png}
	\caption{Temperatura promedio transversal en funcion del tiempo}
	\label{promedio}
\end{figure}

\begin{figure}
	\centering
	\includegraphics[scale=1]{difusion2D.png}
	\caption{Evolucion de T vista en 2 dimensiones}
	\label{2D}
\end{figure}

\begin{figure}
	\centering
	\includegraphics[scale=0.4]{Equilibrio.png}
	\caption{Vista en 2D luego de alcanzar equil. termico}
	\label{2D}
\end{figure}

\begin{figure}
	\centering
	\includegraphics[scale=1]{difusion3D.png}
	\caption{Evolucion de T vista en 3 dimensiones}
	\label{2D}
\end{figure}


\end{document}
